\makeatletter
\@namedef{ver@beamerfontthememetropolis.sty}{9999/99/99}
\makeatother
\usetheme[
%%% 外部主题选项
%    hidetitle,           % 隐藏边栏中的短标题
%    hideauthor,          % 隐藏边栏中的作者缩写
%    hideinstitute,       % 隐藏边栏底部的单位缩写
%    shownavsym,          % 显示导航符号
    width=1.8cm,           % 边栏宽度 (默认是 2 cm)
%    hideothersubsections,% 除了当前section的subsection隐藏其它所有 subsections
%    hideallsubsections,  % 隐藏所有 subsections
    left,               % 边栏位置 (默认在右边)
%%% 颜色主题选项
    %lightheaderbg       % 页眉背景颜色
  ]{NWSUAFsidebar}
  
% 设置字体(参考曾祥东的latex-talk.tex进行设置)
%\usefonttheme{serif,professionalfonts}
\usefonttheme{serif,professionalfonts}
  
% Beamer settings
%\metroset{progressbar=none}
\setbeamerfont{title}{size=\huge, series=\bfseries}
\setbeamerfont*{subtitle}{size=\large, shape=\itshape}
\setbeamerfont{section title}{size=\Large, series=\bfseries}
\setbeamerfont{frametitle}{size=\large, series=\bfseries}
\setbeamerfont{caption}{size=\footnotesize, series=\bfseries}
\setbeamerfont{footnote}{size=\tiny}
\setbeamerfont{alerted text}{series=\bfseries}
%\setbeamertemplate{frame numbering}{\zhnumber[style=Financial]{\insertframenumber}}
\setbeamertemplate{itemize/enumerate subbody begin}{\footnotesize}
\setbeamertemplate{caption}{\parbox{\textwidth}{\centering\insertcaption}\par}
\setbeamertemplate{bibliography item}[text]

% PDF bookmark
\makeatletter
\apptocmd{\beamer@@frametitle}{%
  \only<1>{\expandafter\ifnum\insertcontinuationcount<2\relax
    \bookmark[page=\the\c@page,level=4]{#1}\fi}}{}{}
\makeatother
 
% 部分颜色定义  
\definecolor{Descitem}{RGB}{0, 0, 139}
\definecolor{StdTitle}{RGB}{26, 33, 141}
\definecolor{StdBody}{RGB}{213,24,0}
\definecolor{AlTitle}{RGB}{255, 190, 190}
\definecolor{AlBody}{RGB}{213,24,0}
\definecolor{ExTitle}{RGB}{201, 217, 217}
\definecolor{ExBody}{RGB}{213,24,0}  

% 设置blocks
% Standard block
\setbeamercolor{block title}{fg = Descitem, bg = StdTitle!15!white}
\setbeamercolor{block body}{bg = StdBody!5!white}
% Alert block
\setbeamercolor{block title alerted}{bg = AlTitle}
\setbeamercolor{block body alerted}{bg = AlBody!5!white}
% Example block
\setbeamercolor{block title example}{bg = ExTitle}
\setbeamercolor{block body example}{bg = ExBody!5!white}
\setbeamerfont{block title}{size=\scriptsize}
\setbeamertemplate{blocks}[rounded][shadow=true]

% % 设置列表符号
% \setbeamertemplate{itemize
%   item}{\scriptsize\raise1.25pt\hbox{\donotcoloroutermaths \ding{42}}}%$\blacktriangleright$}}
% \setbeamertemplate{itemize
%   subitem}{\tiny\raise1.5pt\hbox{\donotcoloroutermaths \ding{43}}}%$\blacktriangleright$}}
% \setbeamertemplate{itemize
%   subsubitem}{\tiny\raise1.5pt\hbox{\donotcoloroutermaths \ding{45}}}%$\blacktriangleright$}}
% \setbeamertemplate{enumerate item}{\insertenumlabel.}
% \setbeamertemplate{enumerate subitem}{\insertenumlabel.\insertsubenumlabel}
% \setbeamertemplate{enumerate subsubitem}{\insertenumlabel.\insertsubenumlabel.\insertsubsubenumlabel}
% \setbeamertemplate{enumerate mini template}{\insertenumlabel}

% 设置minted宏包编排代码的参数及用于latex代码排版的简化命令
\setminted{fontsize=\footnotesize, breaklines=true, breakautoindent=false}
\newmintinline{tex}{fontsize=\footnotesize}
\newmintinline{sh}{fontsize=\footnotesize}
\newmintinline[texinlinett]{tex}{escapeinside=||}
\newminted{tex}{fontsize=\scriptsize, bgcolor=yellow!20, frame=lines, autogobble}
\newminted[texcodett]{tex}{autogobble, fontsize=\scriptsize, bgcolor=yellow!20, frame=lines, escapeinside=||}
\newminted[shell]{sh}{autogobble,frame=lines}
\newmintedfile{tex}{bgcolor=yellow!20, fontsize=\footnotesize, frame=lines}

% 改变脚注的字号和符号
\setbeamerfont{footnote}{size=\zihao{7}}
\makeatletter
\def\@fnsymbol#1{\ensuremath{\ifcase#1\or *\or \dagger\or \ddagger\or
   \mathsection\or \mathparagraph\or \|\or **\or \dagger\dagger
   \or \ddagger\ddagger \else\@ctrerr\fi}}
\makeatother
\renewcommand{\thefootnote}{\fnsymbol{footnote}}

\DeclareRobustCommand{\nonumberfootnote}[2][]{%
  \let\thefootnote\relax
  \footnotetext#1{#2}}
% \renewcommand*\footnoterule{} % 取消脚注线

%% 自定义相关的名称宏命令
%% ==================================================
%% \newcommand{\yourcommand}[参数个数]{内容}
% 西北农林科技大学各单位名称
\newcommand{\nwsuaf}{西北农林科技大学}
\newcommand{\cie}{信息工程学院}
\newcommand{\cs}{计算机科学系}

% 定义引号命令
\newcommand{\qtmark}[1]{``#1''}%``''
% 定义带引号的加粗强调命令
\newcommand{\qtb}[1]{\qtmark{\emph{#1}}}
% 定义带引号的加粗加红强调命令
\newcommand{\qtbr}[1]{\qtmark{\emph{\alert{#1}}}}

% 自定义latex简单教程中要用到的宏命令
\newcommand\latex{{ \LaTeX}}%\fontfamily{cmr}\selectfont
\newcommand\msoffice{{\rmfamily MS Office}}
\newcommand\msofficepdf{\texorpdfstring{\msoffice{}}{MS Office}}
\newcommand\wysiwym{\textsc{WYSIWYM}---所想即所得}
\newcommand\wysiwyg{\textsc{WYSIWYG}---所见即所得}
\newcommand\pkg[1]{\texttt{#1}}

\colorlet{msofficecolour}{red!90!white}
\colorlet{latexcolour}{green!90!black}

% % 定义TeXLive的LOGO
\newcommand*\TeXLive{T\kern -.1667em\lower .5ex\hbox {E}\kern
  -.025emX\,Live}
\newcommand\tlive[1][2019]{
  \begin{tikzpicture}[x=1pt,y=1pt,inner sep=0pt,outer sep=0pt]
    \fill [tlblue] (0,0) rectangle (567,160);
    \node [white] at (29.7,33.8) [anchor=south west]
    {\scalebox{10}{\bfseries\TeXLive\~ #1}};
    \node at (388,9) [anchor=south west] {\includegraphics[width=15em]{tl-lion}};
    % \node [anchor=south west] {\includegraphics[height=16em]{logo}};
  \end{tikzpicture}%
}

% 论文模板名称符号
\newcommand{\nwafuprojrep}{%
  \makebox{\rmfamily%
    N\hspace{-0.2ex}\raisebox{-0.5ex}{W}\raisebox{0.5ex}{\hspace{-0.2ex}\textsc{afu}}\hspace{0.3ex}%
    \textsc{Proj}\textsc{Rep}}}          

%% 定义自动扩展垂直间距的命令\stretchon和\stretchoff
%% ==================================================
\def\itemsymbol{$\blacktriangleright$}
\let\svpar\par
\let\svitemize\itemize
\let\svenditemize\enditemize
\let\svitem\item
\let\svcenter\center
\let\svendcenter\endcenter
\let\svcolumn\column
\let\svendcolumn\endcolumn
\def\newitem{\renewcommand\item[1][\itemsymbol]{\vfill\svitem[##1]}}%
\def\newpar{\def\par{\svpar\vfill}}%
\newcommand\stretchon{%
  \newpar%
  \renewcommand\item[1][\itemsymbol]{\svitem[##1]\newitem}%
  \renewenvironment{itemize}%
    {\svitemize}{\svenditemize\newpar\par}%
  \renewenvironment{center}%
    {\svcenter\newpar}{\svendcenter\newpar}%
  \renewenvironment{column}[2]%
    {\svcolumn{##1}\setlength{\parskip}{\columnskip}##2}%
    {\svendcolumn\vspace{\columnskip}}%
}
\newcommand\stretchoff{%
  \let\par\svpar%
  \let\item\svitem%
  \let\itemize\svitemize%
  \let\enditemize\svenditemize%
  \let\center\svcenter%
  \let\endcenter\svendcenter%
  \let\column\svcolumn%
  \let\endcolumn\svendcolumn%
}

%% 签署春秋学期日期命令
\newcommand{\tomonth}{
  \the\year 年\the\month 月
}

\newcommand{\tomonthen}{
  \ifcase\the\month
  \or January%
  \or February%
  \or March%
  \or April%
  \or May%
  \or June%
  \or July%
  \or August%
  \or September%
  \or October%
  \or November%
  \or December%
  \fi, \the\year
}

\newcommand{\tosemester}{
  \the\year 年\ 
  \ifcase\the\month
  \or 秋%1
  \or 春%2
  \or 春%3
  \or 春%4
  \or 春%5
  \or 春%6
  \or 夏%7
  \or 夏%8
  \or 秋%9
  \or 秋%10
  \or 秋%11
  \or 秋%12
  \fi 
}

\newcommand{\tosemesteren}{  
  \ifcase\the\month
  \or Autumn%1
  \or Spring%2
  \or Spring%3
  \or Spring%4
  \or Spring%5
  \or Spring%6
  \or Summer%7
  \or Summer%8
  \or Autumn%9
  \or Autumn%10
  \or Autumn%11
  \or Autumn%12
  \fi, \the\year
}

%% 分栏宽度
\newlength\columnskip
\columnskip 0pt
%% ==================================================          
          
% TiKz绘图设置
 \usetikzlibrary{arrows}
% TikZ宏包扩展
\usetikzlibrary{positioning}
% pgfplots设置
\pgfplotsset{compat=newest,compat/show suggested version=false}

%% 设置绘制目录结构的宏参数
\definecolor{folderbg}{RGB}{124,166,198}
\definecolor{folderborder}{RGB}{110,144,169}
\newlength\Size
\setlength\Size{4pt}
\tikzset{%
  folder/.pic={%
    \filldraw [draw=folderborder, top color=folderbg!50, bottom color=folderbg] (-1.05*\Size,0.2\Size+5pt) rectangle ++(.75*\Size,-0.2\Size-5pt);
    \filldraw [draw=folderborder, top color=folderbg!50, bottom color=folderbg] (-1.15*\Size,-\Size) rectangle (1.15*\Size,\Size);
  },
  file/.pic={%
    \filldraw [draw=folderborder, top color=folderbg!5, bottom color=folderbg!10] (-\Size,.4*\Size+5pt) coordinate (a) |- (\Size,-1.2*\Size) coordinate (b) -- ++(0,1.6*\Size) coordinate (c) -- ++(-5pt,5pt) coordinate (d) -- cycle (d) |- (c) ;
  },
}

% TikZ中的带圈数字
\tikzset{cirnum/.style={midway, circle, fill=black, inner
    sep=0pt, minimum size=3pt, white, yshift=0.35em}}

% 流程图参数设置
\flowchartset{
  free color = green, % 自由连线颜色(默认取green)
  norm color = blue, % 常规连线颜色(默认取blue)
  cong color = red, % 关联连线颜色(默认取red)
  proc fill color = red!10, % 顺序处理框填充颜色(默认取白色)
  proc text width = 4em, % 顺序处理框宽度(默认取8em)
  chain direction = right, % 结点自动布置方向(默认取below)
  minimum node distance = 6mm, % 最小结点间距(默认取6mm)
  maximum node distance = 10mm, % 最大结点间距(默认取60mm)
  flow line width = \pgflinewidth, % 各类流程线线条宽度(默认取当前线条宽度)
  stealth length = 1.5mm, % 箭头长度(默认取1.5mm)
  stealth width = 1.0mm, % 箭头宽度(默认取1.0mm)
}

\forestset{%
  declare autowrapped toks={pic me}{},
  declare boolean register={pic root},
  pic root=0,
  pic dir tree/.style={%
    for tree={%
      folder,
      %font=\ttfamily,
      grow'=0,
      s sep=1.0pt,
      font=\small \sffamily,
      %fit=band,
      %ysep = 1.0pt,
      inner ysep = 2.6pt,
    },
    before typesetting nodes={%
      for tree={%
        edge label+/.option={pic me},
      },
      if pic root={
        tikz+={
          \pic at ([xshift=\Size].west) {folder};
        },
        align={l}
      }{},
    },
  },
  pic me set/.code n args=2{%
    \forestset{%
      #1/.style={%
        inner xsep=2\Size,
        pic me={pic {#2}},
      }
    }
  },
  pic me set={directory}{folder},
  pic me set={file}{file},  
}
%% ==================================================

%%%%%%%%%%%%%%%%%%%%%%%%%%%%%%%%%%%%%%%%%%%%%%%%%%%%%%%%%%%%%%%%%%%%%%
% LaTeX Overlay Generator - Annotated Figures v0.0.2
% Created with http://ff.cx/latex-overlay-generator/
% If this generator saves you time, consider donating 5,- EUR! :-)
%%%%%%%%%%%%%%%%%%%%%%%%%%%%%%%%%%%%%%%%%%%%%%%%%%%%%%%%%%%%%%%%%%%%%%
%                         #1          #2       #3         #4           #5          #6            #7           #8
%\annotatedFigureBox{bottom-left}{top-right}{label}{label-position}{box-color}{label-color}{border-color}{text-color}
\newcommand*\annotatedFigureBoxCustom[8]{\draw[#5,thick,rounded corners] (#1) rectangle (#2);\node at (#4) [fill=#6,thick,shape=circle,draw=#7,inner sep=2pt,font=\sffamily,text=#8] {\textbf{#3}};}
\newcommand*\annotatedFigureBoxLabel[4]{\annotatedFigureBoxCustom{#1}{#2}{#3}{#4}{red}{white}{black}{black}}
\newcommand*\annotatedFigureBox[3]{\draw[#3,thick,rounded corners=0.5mm] (#1) rectangle (#2);}
\newenvironment {annotatedFigure}[1]{\centering\begin{tikzpicture}\node[anchor=south west,inner sep=0] (image) at (0,0) { #1};\begin{scope}[x={(image.south east)},y={(image.north west)}]}{\end{scope}\end{tikzpicture}}
%%%%%%%%%%%%%%%%%%%%%%%%%%%%%%%%%%%%%%%%%%%%%%%%%%%%%%%%%%%%%%%%%%%%%%

% 动态改变menukeys宏包的win/mac样式
\makeatletter
\def\setmenukeyswin{\def\tw@mk@os{win}}
\def\setmenukeysmac{\def\tw@mk@os{mac}}
\makeatother

% 插图路径设置
% ==================================================
\graphicspath{{figures/}}%图片所在的目录
% ==================================================

% 为标题页指定一个 logo
\pgfdeclareimage[height=0.8cm]{titlepagelogo}{nwsuaflogo/nwsuaf_logo_new}% 标题页
\titlegraphic{% 标题页底部
  \pgfuseimage{titlepagelogo}%
}

\newcommand\CASE[1]{{\addfontfeatures{Letters=Uppercase}#1}}
\newcommand\jatext[1]{{\addCJKfontfeatures{Language=Japanese}#1}}
\newcommand\link[1]{\href{#1}{\ \faExternalLinkSquare*}}%\faLink \faExternalLink*
\newcommand\usv[1]{\texttt{U+#1}}
\newCJKfontfamily\SourceHanSerif{Source Han Serif SC SemiBold}

% PoZheHao, see https://github.com/CTeX-org/ctex-kit/issues/382
\ExplSyntaxOn
\xeCJK_new_class:n { PoZheHao }
\__xeCJK_save_CJK_class:n { PoZheHao }
\xeCJK_declare_char_class:nn { PoZheHao } { "2014 }
\seq_map_inline:Nn \g__xeCJK_class_seq
  {
    \str_if_eq:nnF {#1} { PoZheHao }
      {
        \xeCJK_copy_inter_class_toks:nnnn { PoZheHao } {#1} { FullRight } {#1}
        \xeCJK_copy_inter_class_toks:nnnn {#1} { PoZheHao } {#1} { FullRight }
      }
  }
\ExplSyntaxOff

% Hack
% Use small caps for LaTeX symbol
\DeclareRobustCommand{\LaTeX}{%
  L\kern-.3em%
  \raisebox{.2em}{\textsc{a}}\kern-.14em%
  \TeX}
% Compatibility with unicode-math
\DeclareRobustCommand{\LaTeXe}{%
  \LaTeX\kern.15em2%
  \hbox{%
    \if b\expandafter\@car\f@series\@nil
      $_{\textstyle\symbf{\varepsilon}}$%
    \else
      $_{\textstyle\varepsilon}$%
      \fi}
  }  



%%% Local Variables: 
%%% mode: latex
%%% TeX-master: "../main.tex"
%%% End: 

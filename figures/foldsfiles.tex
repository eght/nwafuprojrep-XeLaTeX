%  \begin{center}
    % \includegraphics[height=0.8\textheight]{coworks}
%    \scalebox{0.5}{
      \begin{forest}
        pic dir tree,
        pic root,
        for tree={% folder icons by default; override using file for file icons
          directory,
        },
        [jobname【工作根目录】%, 
          [bib【参考文献数据库目录,根据需要,可以有多个数据库】%
            [sample.bib【样例数据库,根据需要,可以有多个数据库】, file%
            ]   
          ]
          [codes【报告中需要的代码源文件,可以有多个,根据需要增减】        
            [ex04-01.cpp, file
            ]
            [$\vdots$, file
            ]
          ]
          [figs【插图目录,可根据需要增减】
            [plot
            ]
            [xxxx.png, file
            ]
            [xxxx.pdf, file
            ]
          ]
          [settings【自定义命令、环境等文件、引入宏包文件,可根据需要进行调整】        
            [format.tex【自定义命令、环境、参数设置等】, file
            ]
            [math-commands.tex【自定义数学符号命令】, file
            ]
            [packages.tex【需要引入的宏包】, file
            ]
            [terms.tex【自定义术语命令】, file
            ]
          ]
          [boxie.sty【盒子宏包,如果需要则置于根目录,并在settings/package.tex中引用该宏包】, file
          ]
          [fvextra.sty【子宏包需要的宏包,如果使用了boxie宏包,则必须置于根目录】, file
          ]
          [lstlinebgrd.sty【子宏包需要的宏包,如果使用了boxie宏包,则必须置于根目录】, file
          ]
          [main.tex【主控文件,\emph{必须存在},且置于根目录】, file
          ]
          [Makefile【ake命令需要的脚本文件,如能执行make,可以在根目录执行make命令进行编译】, file
          ]
          [nwafuprojrep.cls【文档类文件,即模板文件,必须存在,且置于根目录】, file
          ]
          [pgf-umlcd.sty【UML图绘制宏包,如果需要则置于根目录,并在settings/package.tex中引用该宏包】, file
          ]
          [tikz-flowchart.sty【流程图绘制宏包,如果需要则置于根目录,并在settings/package.tex中引用该宏包】, file
          ]
          [tikz-imglabels.sty【图像标注/标记宏包,如果需要则置于根目录,并在settings/package.tex中引用该宏包】, file
          ]
          [.latexmkrc【latexmk命令需要的脚本文件,如能执行latexmk,可以在根目录执行latexmk命令进行编译】, file
          ]
        ]
      \end{forest}
%    }
%  \end{center}
